\refstepcounter{appendixctr}\label{hwansappendix}%
\appendix\chapter{Appendix \ref{hwansappendix}: Hints and Solutions}
	
%==================================================================
%==================================================================
%========================= Solutions ==============================
%==================================================================
%==================================================================

\hwanssection{Solutions to Selected Homework Problems}

\beginsolutions{6}

\hwsolnhdr{carousel-singularities} 
(a) There are singularities at $r=0$, where $g_{\theta'\theta'}=0$, and $r=1/\omega$,
where $g_{tt}=0$. These are considered singularities because the inverse of the metric
blows up. They're coordinate singularities, because they can be removed by a change of
coordinates back to the original non-rotating frame. (b) This one has singularities in
the same places. The one at $r=0$ is a coordinate singularity, because at small $r$
the $\omega$ dependence is negligible, and the metric is simply that of ordinary
plane polar coordinates in flat space. The one at $r=1/\omega$ is not a coordinate
singularity. The following Maxima code calculates its scalar curvature $R=R\indices{^a_a}$,
which is esentially just the Gaussian curvature, since this is a two-dimensional space.
\begin{listing}{1}
load(ctensor);
dim:2;
ct_coords:[r,theta];
lg:matrix([-1,0],
          [0,-r^2/(1-w^2*r^2)]);
cmetric();   
ricci(true);
scurvature();
\end{listing}
% carousel-singularities.mac
The result is $R=6\omega^2/(1-2\omega^2r^2+\omega^4r^4)$.
This blows up at $r=1/\omega$, which shows that this is not a coordinate
singularity. The fact that $R$ does not blow up at $r=0$ is consistent with our
earlier conclusion that $r=0$ is a coordinate singularity, but would not have been
sufficient to prove that conclusion.
