\refstepcounter{appendixctr}\label{hwansappendix}%
\appendix\noindent\formatlikechapter{Appendix \ref{hwansappendix}: Hints and solutions}
	
%==================================================================
%==================================================================
%========================= Hints ==============================
%==================================================================
%==================================================================

\hwanssection{Hints}

\noindent\formatlikesubsection{Hints for Chapter 1}

\hwsolnhdr{tossed-clock} Apply the equivalence principle.\label{hint:tossed-clock}

%==================================================================
%==================================================================
%========================= Solutions ==============================
%==================================================================
%==================================================================

\hwanssection{Solutions to Selected Homework Problems}

\beginsolutions{1}

\hwsolnhdr{ordered-geom-finite-models}

Pick two points P1 and P2. By O2, there is another point P3 that is distinct
from P1 and P2. (Recall that the notation [ABC] was defined so that all three
points must be distinct.) Applying O2 again, there must be a further point
P4 out beyond P3, and by O3 this can't be the same as P1. Continuing in this
way, we can produce as many points as there are integers.

\hwsolnhdr{have-spacesuit}

(a) If the violation of (1) is tiny, then of course Kip won't really have any
practical way to violate (2), but the idea here is just to illustrate the
idea, so to make things easy, let's imagine an unrealistically large violation
of (1). Suppose that neutrons have about the same inertial mass as protons, but
zero gravitational mass, in extreme violation of (1). This implies that neutron-rich
elements like uranium would have a much lower gravitational acceleration on earth
than ones like oxygen that are roughly 50-50 mixtures of neutrons and protons.
Let's also simplify by making a second unrealistically extreme assumption: let's
say  that Kip has a keychain in his pocket made of neutronium, a substance composed of
pure neutrons. On earth, the keychain hovers in mid-air. Now he can release
his keychain in the prison cell. If he's on a planet, it will hover.
If he's in an accelerating spaceship, then the keychain will follow Newton's
first law (its tendency to do so being measured by its nonzero inertial mass),
while the deck of the ship accelerates up to hit it.

(b) It violates O1. O1 says that objects prepared in identical inertial states
(as defined by two successive events in their motion) are predicted to have
identical motion in the future. This fails in the case where Kip releases the
neutronium keychain side by side with a penny.

\hwsolnhdr{tossed-clock}
By the equivalence principle, we can adopt a frame tied to the tossed clock, B, and in this
frame there is no gravitational field. We see a desk and clock A go by. The desk applies
a force to clock A, decelerating it and then reaccelerating it so that it comes back.
We've already established that the effect of motion is to slow down time, so clock
A reads a smaller time interval.

\hwsolnhdr{ep-charge}
(a) In case 1 there is no source of energy, so the particle cannot radiate.
In case 2-4, the particle radiates, because there are sources of energy (loss of
gravitational energy in 2 and 3, the rocket fuel in 4).

(b) In 1, Newton says the object is subject to zero net force, so its motion
is inertial. In 2-4, he says the object is subject to a nonvanishing net force,
so its motion is noninertial. This matches up with the results of the energy analysis.

(c) The equivalence principle, as discussed on page \pageref{sec:chiao-paradox},
is vague, and is particularly difficult to apply successful and unambiguously to
situations involving electrically charged objects, due to the difficulty of
defining locality. Applying the equivalence principle in the most naive way,
we predict that there can be no radiation in cases 2 and 3 (because the object is
following a geodesic, minding its own business).
In case 4, everyone agrees that there will be radiation observable back on earth
(although it's possible that it would not be observable to an observer momentarily
matching velocities with the rocket).
The naive equivalence principle says that 1 and 4 must give the same result, so
we should have radiation in 1 as well. These predictions are wrong in two out of
the four equations, which tells us that we had better either not apply the equivalence
principle to charged objects, or not apply it in such a naive way.

\hwsolnhdr{dewitt-estimate}

(a) The dominant form of radiation from the orbiting charge will be the lowest-order
nonvanishing multipole, which in this case is a dipole. The power radiated from
a dipole scales like $d^2\omega^4$, where $d$ is the dipole moment. For an orbit of
radius $r$, this becomes $q^2r^2\omega^4$. To find the reaction force on the charged particle,
we can use the relation $p=E/c$ for electromagnetic waves (section \ref{sec:gravitational-redshifts}),
which tells us that the force is equal to the power, up to a proportionality constant $c$.
Therefore $a_r\propto q^2r^2\omega^4/m$. The gravitational acceleration is $a_g=\omega^2 r$,
so we have $a_r/a_g \propto (q^2/m)\omega^2 r$, or $a_r/a_g \propto (q^2/m)a_g$, where the
$a_g$ on the right can be taken as an orbital parameter, and for a low-earth orbit is very nearly equal to
the usual acceleration of gravity at the earth's surface.

(b) In SI units, $a_r/a_g \sim (k/c^4)(q^2/m)a_g$, where $k$ is the Coulomb constant.

(c) The result is $10^{-34}$. If one tried to do this experiment in reality, the effect would be
impossible to detect, because the proton would be affected much more strongly by ambient electric and
magnetic fields than by the effect we've calculated.

Remark: It is odd that the result depends on $q^2/m$, rather than on the charge-to-mass ratio $q/m$,
as is usually the case for a test particle's trajectory. This means that we get a different answer
if we take two identical objects, place them side by side, and consider them as one big object! This is not as unphysical
as it sounds. The two side-by-side objects radiate coherently, so the field they radiate is doubled, and the
radiated power is quadrupled. Each object's rate of orbital decay is doubled, with the extra effect coming
from electromagnetic interactions with the other object's fields.

\beginsolutions{2}

\hwsolnhdr{clock-postulate}

(a) Let $t$ be the time taken in the lab frame for the light to go from one mirror to the other,
and $t'$ the corresponding interval in the clock's frame. Then $t'=L$, and $(vt)^2+L^2=t^2$,
where the use of the same $L$ in both equations makes use of our prior knowledge that there
is no transverse length contraction.
Eliminating $L$, we find the expected expression for $\gamma$, which is independent of $L$
(b) If the result of a were independent of $L$, then the relativistic time dilation would depend
on the details of the construction of the clock measuring the time dilation. We would be forced
to abandon the geometrical interpretation of special relativity.
(c) The effect is to replace $vt$ with $vt+at^2/2$ as the quantity inside the parentheses
in the expression $(\ldots)^2+L^2=t^2$. The resulting correction terms are of higher order in
$t$ than the ones appearing in the original expression, and can therefore be made as small
in relative size as desired by shortening the time $t$. But this is exactly what happens when
we make the clock sufficiently small.

\hwsolnhdr{sagnac-area}

(a) Reinterpret figure \figref{thomas-as-area} on p.~\pageref{fig:thomas-as-area} as a picture of a Sagnac
ring interferometer. Let light waves 1 and 2 move around the loop in opposite senses. Wave 1 takes time
$t_{1i}$ to move inward along the crack, and time $t_{1o}$ to come back out. Wave 2 takes times
$t_{2i}$ and $t_{2o}$. But $t_{1i}=t_{2i}$ (since the two world-lines are identical), and similarly
$t_{1o}=t_{2o}$. Therefore creating the crack has no effect on the interference between 1 and 2,
and splitting the big loop into two smaller loops merely splits the total phase shift between them.
(b) For a circular loop of radius $r$, the time of flight of each wave is proportional to $r$, and
in this time, each point on the circumference of the rotating interferometer travels a distance
$v(\text{time})=(\omega r)(\text{time})\propto r^2$. (c) The effect is proportional to area, and
the area is zero. (d) The light clock in c has its two ends synchronized according to the Einstein
prescription, and the success of this synchronization verifies Einstein's assumption of commutativity
in this particular case. If we make a Sagnac interferometer in the shape of a triangle, then the Sagnac
effect measures the failure of Einstein's assumption that all three corners can be synchronized
with one another.

\hwsolnhdr{velocity-addition-matrix-taylor}

Here is the program:
\begin{listing}{1}
L1:matrix([cosh(h1),sinh(h1)],[sinh(h1),cosh(h1)]);
L2:matrix([cosh(h2),sinh(h2)],[sinh(h2),cosh(h2)]);
T:L1.L2;
taylor(taylor(T,h1,0,2),h2,0,2);
\end{listing}
The diagonal components of the result are both $1+\eta_1^2/2+\eta_2^2/2+\eta_1\eta_2+\ldots$
Everything after the 1 is nonclassical. The 
off-diagonal components are $\eta_1+\eta_2+\eta_1\eta_2^2/2+\eta_2\eta_1^2/2+\ldots$,
with the third-order terms being nonclassical.

\beginsolutions{3}

\hwsolnhdr{cylinder-spacetime}

(a) As discussed in example \ref{eg:extrinsic-curvature} on page \pageref{eg:extrinsic-curvature}, a cylinder has local,
intrinsic properties identical to those of flat space. The cylindrical model therefore has the same properties L1-L5
as our standard model of Lorentzian space, provided that L1-L5 are taken as purely local statements.

(b) The cylindrical model does violate L3. In this model, the doubly-intersecting world-lines
described by property G will not occur if the world-lines are oriented exactly parallel to the cylinder. This picks
out a preferred direction in space, violating L3 if L3 is interpreted globally. Frames moving parallel to the axis have
different properties from frames moving perpendicular to the axis.

But just because this particular model violates the
global interpretation of L3, that doesn't mean that all models of G violate it. We could instead construct a model in
which space wraps around in every direction. In the 2+1-dimensional case, we can visualize the spatial part of such a model as the surface
of a doughnut embedded in three-space, with the caveat that we don't want to think of the doughnut hole's circumference as being
shorter than the doughnut's outer radius. Giving up the idea of a visualizable model embedded in a higher-dimensional space,
we can simply take a three-dimensional cube and identify its opposite faces. Does this model violate L3? It's not quite as
obvious, but actually it does. The spacelike great-circle geodesics of this model come in different circumferences, with the shortest
being those parallel to the cube's axes.

We can't prove by constructing a finite number of models that every possible model of G violates L3. The two models we've found, however,
can make us suspect that this is true, and can give us insight into how to prove it. For any pair of world-lines that
provide an example of G, we can fix a coordinate system K in which the two particles started out at A by flying off back-to-back.
In this coordinate system, we can measure the sum of the distances traversed by the two particles from A to B. (If homegeneity, L1,
holds, then they make equal contributions to this sum.) The fact that the world-lines were traversed by material particles
means that we can, at least in principle, visit every point on them and measure the total distance using rigid rulers.
We call this the circumference of the great circle AB, as measured in a particular frame.
The set of all such circumferences has some greatest lower bound. If this bound is zero, then such geodesics can exist locally, and
this would violate even the local interpretation of L1-L5. If the bound is nonzero, then let's fix a circle that has
this minimum circumference. Mark the spatial points this circle passes through, in the frame of reference defined above.
This set of points is a spacelike circle of minimum radius. Near a given point on the circle, the circle looks like
a perfectly straight axis, whose orientation is presumably random. Now let some observer $\zu{K}'$ travel around this circle at a velocity
$v$ relative to K, measuring the circumference with a Lorentz-contracted ruler. The circumference is greater than the minimal
one measured by K. Therefore for any axis with a randomly chosen orientation, we have a preferred rest frame in which the
corresponding great circle has minimum circumference. This violates L3. Thanks to physicsforums user atyy for suggesting this
argument.

More detailed discussions of these issues are given in Bansal et al., \url{arxiv.org/abs/gr-qc/0503070v1},
and Barrow and Levin, \url{arxiv.org/abs/gr-qc/0101014v1}.

\hwsolnhdr{gps-t-discontinuity}

The coordinate $T$ would have a discontinuity of $2\pi\omega r^2/(1-\omega^2 r^2)$. Reinserting factors of $c$ to make it work
out in SI units, we have $2\pi\omega r^2 c^{-2}/(1-\omega^2 r^2 c^{-2}) \approx 207\ \zu{ns}$. The exact error in position that
would result is dependent on the geometry of the current position of the satellites, but it would be on the order of $c\Delta T$,
which is $\sim 100\ \munit$. This is considerably worse than civilian GPS's 20-meter error bars.

\hwsolnhdr{carousel-paradox}

The process that led from the Euclidean metric of example \ref{eg:metric-in-polar-coords} on page \pageref{eg:metric-in-polar-coords}
to the non-Euclidean one of equation [\ref{eq:rotating-spatial}] on page \pageref{eq:rotating-spatial} was not just a series
of coordinate transformations. At the final step, we got rid of the variable $t$, reducing the number of dimensions by one.
Similarly, we could take a Euclidean three-dimensional space and eliminate all the points except for the ones on the surface
of the unit sphere; the geometry of the embedded sphere is non-Euclidean, because we've redefined geodesics to be lines that
are ``as straight as they can be'' (i.e., have minimum length) while restricted to the sphere. In the example of the carousel, the final step effectively
redefines geodesics so that they have minimal length as determined by a chain of radar measurements.

\hwsolnhdr{carousel-metric}

(a) No.
The track is straight in the lab frame, but curved in the
rotating frame. Since the spatial metric in the rotating frame is symmetric with respect to clockwise and counterclockwise,
the metric can never result in geodesics with a specific handedness.
(b) The $\der\theta'^2$ term of the metric blows up here. A geodesic connecting point A, at $r=1/\omega$, with point B, at $r<1/\omega$,
must have minimum length. This requires that the geodesic be directly radial at A, so that $\der\theta'=0$; for if not, then we could
vary the curve slightly so as to reduce $|\der\theta'|$, and the resulting increase in the $\der r^2$ term would be negligible
compared to the decrease in the $\der\theta'^2$ term. (c) No. As we found in part a, laser beams can't be used to form geodesics.

\hwsolnhdr{penrose-generalization}
A and B are equivalent under a Lorentz transformation, so the Penrose result clearly includes B. The outline of the sphere is still spherical.
C is also equivalent to A and B, because there are only two effects (Lorentz contraction and optical aberration), and both of
them depend only on the observer's instantaneous velocity, not on his history of motion.
D is not a well-defined question. When asking this question, we're implicitly assuming that the sphere has some
``real'' shape, which appears different because the sphere has been set into motion. But
you can't impart an angular acceleration to a perfectly rigid body in relativity.

\beginsolutions{4}

\hwsolnhdr{lhc-proton-speed}

To avoid loss of precision in numerical operations like subtracting $v$ from $1$,
it's better to derive an ultrarelativistic approximation. The velocity corresponding
to a given $\gamma$ is $v=\sqrt{1-\gamma^{-2}}\approx 1-1/2\gamma^2$, so
$1-v\approx 1/2\gamma^2=(m/E)^2/2$. Reinserting factors of $c$ so as to make the units
come out right in the SI system, this becomes $(mc^2/E)^2/2=9\times 10^{-9}$.

\hwsolnhdr{hafele-frame-indept}

The time on the clock is given by $s=\int \der s$, where the integral is over the clock's
world-line. The quantity $\der s$ is our prototypical Lorentz scalar, so it's frame-independent.
An integral is just a sum, and the tensor transformation laws are linear, so the integral of
a Lorentz scalar is still a Lorentz scalar. Therefore $s$ is frame-independent. There is
no requirement that we use an inertial frame. It would also work fine, for example, in a frame
rotating with the earth. We don't even need to have a frame of reference.
All of the above applies equally well to any
coordinate system at all, even one that doesn't have any sensible interpretation as some
observer's frame of reference.

\hwsolnhdr{dirac-sea-invariant}

Such a transformation would take an energy-momentum four-vector $(E,\vc{p})$, with $E>0$, to
a different four-vector $(E',\vc{p}')$, with $E'<0$. That transformation would also have the
effect of transforming a timelike displacement vector from the future light cone to the past
light cone. But the Lorentz transformations were specifically constructed so as to preserve
causality (property L5 on p.~\pageref{sec:lorentz-geometry}), so this can't happen.

\hwsolnhdr{doppler-three-d}

A spatial plane is determined by the light's direction of propagation and the relative velocity
of the source and observer, so the 3+1 case reduces without loss of generality to 2+1 dimensions.
The frequency four-vector must be lightlike, so its most general possible form is
$(f,f\cos\theta,f\sin\theta)$, where $\theta$ is interpreted as the angle between the direction of
propagation and the relative velocity. Putting this through a Lorentz boost along the $x$ axis,
we find $f'=\gamma f(1+v\cos\theta)$, which agrees with Einstein's equation on page
\pageref{einstein-doppler}, except for the arbitrary convention involved in defining the sign of $v$.

\hwsolnhdr{classical-electron-radius}

The exact result depends on how one assumes the charge is distributed, so this can't be any more than a rough
estimate. The energy density is $(1/8\pi k)E^2 \sim ke^2/r^4$, so the total energy is an integral of the form
$\int r^{-4} \der V \sim \int r^{-2} \der r$, which diverges like $1/r$ as the lower limit of integration approaches
zero. This tells us that most of the energy is at small values of $r$, so to a rough approximation we can just
take the volume of integration to be $r^3$ and multiply by a fixed energy density of $ke^2/r^4$. This gives
an energy of $\sim ke^2/r$. Setting this equal to $mc^2$ and solving for $r$, we find $r \sim ke^2/mc^2 \sim 10^{-15}\ \munit$.

Remark: Since experiments have shown that electrons do \emph{not} have internal structure on this scale, we conclude that
quantum-mechanical effects must prevent the energy from blowing up as $r\rightarrow 0$.

\beginsolutions{5}

\hwsolnhdr{uniform-field}

(a) Expanding in a Taylor series, they both have $g_{tt}=1+2gz+\ldots$

(b) This property holds for [\ref{eq:uniform-field-metric-rindler}] automatically because of the way it was constructed.
In [\ref{eq:uniform-field-metric-exp}], the nonvanishing Christoffel
symbols (ignoring permutations of the lower indices) are $\Gamma\indices{^t_{zt}}=g$ and $\Gamma\indices{^z_{tt}}=ge^{2gz}$.
We can apply the geodesic equation with the affine parameter taken to be the proper time, and this gives 
$\ddot{z}=-ge^{2gz}\dot{t}^2$, where dots represent differentiation with respect to proper time.
For a particle instantaneously at rest, $\dot{t}=1/\sqrt{g_{tt}}=e^{-2gz}$, so $\ddot{z}=-g$.

(c) [\ref{eq:uniform-field-metric-rindler}] was constructed by performing a change of coordinates on a flat-space metric, so it is flat. 
The Riemann tensor of [\ref{eq:uniform-field-metric-exp}] has $R\indices{^t_{ztz}}=-g^2$, so [\ref{eq:uniform-field-metric-exp}] isn't flat.
Therefore the two can't be the same under a change of coordinates.

(d) [\ref{eq:uniform-field-metric-rindler}] is flat, so its curvature is constant. [\ref{eq:uniform-field-metric-exp}] has the
property that under the transformation $z\rightarrow z+c$, where $c$ is a constant, the only change is a rescaling of the
time coordinate; by coordinate invariance, such a rescaling is unobservable.

\beginsolutions{6}

\hwsolnhdr{carousel-singularities} 
(a) There are singularities at $r=0$, where $g_{\theta'\theta'}=0$, and $r=1/\omega$,
where $g_{tt}=0$. These are considered singularities because the inverse of the metric
blows up. They're coordinate singularities, because they can be removed by a change of
coordinates back to the original non-rotating frame.\\
(b) This one has singularities in
the same places. The one at $r=0$ is a coordinate singularity, because at small $r$
the $\omega$ dependence is negligible, and the metric is simply that of ordinary
plane polar coordinates in flat space. The one at $r=1/\omega$ is not a coordinate
singularity. The following Maxima code calculates its scalar curvature $R=R\indices{^a_a}$,
which is esentially just the Gaussian curvature, since this is a two-dimensional space.
\begin{listing}{1}
load(ctensor);
dim:2;
ct_coords:[r,theta];
lg:matrix([-1,0],
          [0,-r^2/(1-w^2*r^2)]);
cmetric();   
ricci(true);
scurvature();
\end{listing}
% carousel-singularities.mac
The result is $R=6\omega^2/(1-2\omega^2r^2+\omega^4r^4)$.
This blows up at $r=1/\omega$, which shows that this is not a coordinate
singularity. The fact that $R$ does not blow up at $r=0$ is consistent with our
earlier conclusion that $r=0$ is a coordinate singularity, but would not have been
sufficient to prove that conclusion.\\
(c) The argument is incorrect. The Gaussian curvature is not just proportional to
the angular deficit $\epsilon$, it is proportional to the
limit of $\epsilon/A$, where $A$ is the area of the triangle. The area of the triangle
can be small, so there is no upper bound on the ratio $\epsilon/A$.
Debunking the argument restores consistency with the answer to part b.

\beginsolutions{7}

\hwsolnhdr{cosmology-killing}
Of course the universe is lumpy, so strictly speaking it has no Killing vectors at all, but we're speaking
of the symmetries of a model, not the real universe. If the universe is homogeneous, then we expect it to
have three Killing vectors, corresponding to the three spatial dimensions. Since the Big Bang had different
properties than later times, we expect that there will be no timelike Killing vector, and it is not stationary
or static.

\hwsolnhdr{petrov-strength}

Under these special conditions, the geodesic equations become $\ddot{r}=\Gamma\indices{^r_{tt}}\dot{t}^2$,
$\ddot{\phi}=0$, $\ddot{t}=0$,
where the dots can in principle represent differentation with respect to any affine parameter we like, but
we intend to use the proper time $s$.
By symmetry, there will be no motion in the $z$ direction.
The Christoffel symbol equals $-(1/2)e^r(\cos\sqrt{3}r-\sqrt{3}\sin\sqrt{3}r)$. At a location where
the cosine equals 1, this is simply $-e^r/2$. For $\dot{t}$, we have $\der t/\der s=1/\sqrt{g_{tt}}=e^{-r/2}$.
The result of the calculation is simply $\ddot{r}=-1/2$, which is independent of $r$.

\hwsolnhdr{rotation-with-no-center}

The Petrov metric is one example. The metric has no singularities anywhere, so the $r$ coordinate
can be extended from $-\infty$ to $+\infty$, and there is no point that can be considered the center.
The existence of a $\der\phi \der t$ term in the metric shows that it is not static.

A simpler example is a spacetime made by taking a flat Lorentzian space and making it wrap around topologically
into a cylinder, as in problem \ref{hw:cylinder-spacetime} on p.~\pageref{hw:cylinder-spacetime}. As discussed
in the solution to that problem, this spacetime has a preferred state of rest in the azimuthal direction.
In a frame that is moving azimuthally relative to this state of rest, the Lorentz transformation requires that
the phase of clocks be adjusted linearly as a function of the azimuthal coordinate $\phi$. As described
in section \ref{sec:carousel}, this will cause a discontinuity once we wrap around by $2\pi$, and therefore
clock synchronization fails, and this frame is not static.

\beginsolutions{8}

\hwsolnhdr{field-equations-parity}
The cosmological constant is a scalar, so it doesn't change under reflection. The metric is also invariant
under reflection of any coordinate. This follows because we have assumed that the coordinates are locally Lorentzian,
so that the metric is diagonal. It can therefore be written as a line element in which the differentials are all
squared. This establishes that the $\Lambda g_{ab}$ is invariant under any spatial or temporal reflection.

The specialized form of the energy-momentum tensor $diag(-\rho,P,P,P)$ is also clearly invariant under
any reflection, since both pressure and mass-energy density are scalars.

The form of the tensor transformation law for a rank-2 tensor guarantees that the diagonal elements of such
a tensor stay the same under a reflection. The off-diagonal elements will flip sign, but since only the
$G$ and $T$ terms in the field equation have off-diagonal terms, the field equations remain valid under
reflection.

In summary, the Einstein field equations retain the same form under reflection in any coordinate. This
important symmetry property, which is part of the  Poincar\'{e} group\index{Poincar\'{e} group} in special relativity,
is retained when we make the transition to general relativity. It's a discrete symmetry, so it wasn't guaranteed
to exist simply because of general covariance, which relates to continuous coordinate transformations.

\hwsolnhdr{uniform-field-sources}

(a) The Ricci tensor is $R_{tt}=g^2e^{2gz}$, $R_{zz}=-g^2$. The scalar curvature is $2g^2$, which is constant, as expected.

(b) Both $G_{tt}$ and $G_{zz}$ vanish by a straightforward computation.

(c) The Einstein tensor is $G_{tt}=0$, $G_{xx}=G_{yy}=g^2$, $G_{zz}=0$. It is unphysical because it has a zero mass-energy density,
but a nonvanishing pressure.

\beginsolutions{9}

\hwsolnhdr{circular-radiation}
Since the orbits are initially circular, we find by symmetry that they must remain circular as
they evolve. The only possible relativistic effect is that the circles shrink or expand; all
we need to do is figure out which.
Each body feels a force from the other body that originated at an earlier point in its orbit.
A sketch shows that this force has a forward component, so we would think that it would do
positive work. But we need to be careful about how we interpret the expression $W=\int\vc{F}\cdot\der\vc{r}$
for mechanical work. In the case of a non-contact force, the work-kinetic energy theorem says that
$W$ is to be interpreted as the change in the \emph{kinetic} energy of the object acted upon, not its
total energy. By Kepler's laws, a greater kinetic energy exists when the radius of the circular orbit
is smaller. The orbits shrink, and the total energy (kinetic plus potential) decreases with time.
As in the elliptical-orbit case, we find that the system loses energy, and we interpret this as being
due to gravitational radiation that carries the energy away.

\hwsolnhdr{why-geodesics-for-test-particles}
(a) The members of the Hulse-Taylor system are spiraling toward one another as they lose energy to
gravitational radiation. If one of them were replaced with a low-mass test particle, there
would be negligible radiation, and the motion would no longer be a spiral. This is similar
to the issues encountered on pp.~\pageref{sec:chiao-paradox}ff because the neutron stars in
the Hulse-Taylor system suffer a back-reaction from their own gravitational radiation.

(b) If this occurred, then the particle's world-line would be displaced in space relative
to a geodesic of the spacetime that would have existed without the presence of the particle.
What would determine the direction of that displacement?
It can't be determined by properties of this preexisting, ambient spacetime, because
the Riemann tensor is that spacetime's only local, intrinsic, observable property.
At a fixed point in spacetime, the Riemann tensor is even under spatial reflection, so there's no way it can distinguish
a certain direction in space from the opposite direction.

What else could determine this mysterious displacement?
By assumption, it's not determined by a preexisting, ambient electromagnetic field.
If the particle had charge, the direction could be one imposed by the back-reaction from the electromagnetic
radiation it had emitted in the past. If the particle had a lot of mass, then we could have
something similar with gravitational radiation, or some other nonlinear interaction of the
particle's gravitational field with the ambient field. But these nonlinear or back-reaction
effects are proportional to $q^2$ and $m^2$, so they vanish when $q=0$ and $m\rightarrow 0$.

The only remaining possibility is that the result violates the symmetry of space expressed by L1 on p.~\pageref{sec:lorentz-geometry};
the Lorentzian geometry is the result of L1-L5, so violating L1 should be considered a violation of Lorentz
invariance.
