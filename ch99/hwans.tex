\refstepcounter{appendixctr}\label{hwansappendix}%
\appendix\noindent\formatlikechapter{Appendix \ref{hwansappendix}: Hints and solutions}
	
%==================================================================
%==================================================================
%========================= Hints ==============================
%==================================================================
%==================================================================

\hwanssection{Hints}

\noindent\formatlikesubsection{Hints for Chapter 1}

\hwsolnhdr{tossed-clock} Apply the equivalence principle.\label{hint:tossed-clock}

%==================================================================
%==================================================================
%========================= Solutions ==============================
%==================================================================
%==================================================================

\hwanssection{Solutions to Selected Homework Problems}

\beginsolutions{1}

\hwsolnhdr{ordered-geom-finite-models}

Pick two points P1 and P2. By O2, there is another point P3 that is distinct
from P1 and P2. (Recall that the notation [ABC] was defined so that all three
points must be distinct.) Applying O2 again, there must be a further point
P4 out beyond P3, and by O3 this can't be the same as P1. Continuing in this
way, we can produce as many points as there are integers.

\hwsolnhdr{have-spacesuit}

(a) If the violation of (1) is tiny, then of course Kip won't really have any
practical way to violate (2), but the idea here is just to illustrate the
idea, so to make things easy, let's imagine an unrealistically large violation
of (1). Suppose that neutrons have about the same inertial mass as protons, but
zero gravitational mass, in extreme violation of (1). This implies that neutron-rich
elements like uranium would have a much lower gravitational acceleration on earth
than ones like oxygen that are roughly 50-50 mixtures of neutrons and protons.
Let's also simplify by making a second unrealistically extreme assumption: let's
say  that Kip has a keychain in his pocket made of neutronium, a substance composed of
pure neutrons. On earth, the keychain hovers in mid-air. Now he can release
his keychain in the prison cell. If he's on a planet, it will hover.
If he's in an accelerating spaceship, then the keychain will follow Newton's
first law (its tendency to do so being measured by its nonzero inertial mass),
while the deck of the ship accelerates up to hit it.

(b) It violates O1. O1 says that objects prepared in identical inertial states
(as defined by two successive events in their motion) are predicted to have
identical motion in the future. This fails in the case where Kip releases the
neutronium keychain side by side with a penny.

\hwsolnhdr{tossed-clock}
By the equivalence principle, we can adopt a frame tied to the tossed clock, B, and in this
frame there is no gravitational field. We see a desk and clock A go by. The desk applies
a force to clock A, decelerating it and then reaccelerating it so that it comes back.
We've already established that the effect of motion is to slow down time, so clock
A reads a smaller time interval.

\hwsolnhdr{ep-charge}
(a) In case 1 there is no source of energy, so the particle cannot radiate.
In case 2-4, the particle radiates, because there are sources of energy (loss of
gravitational energy in 2 and 3, the rocket fuel in 4).

(b) In 1, Newton says the object is subject to zero net force, so its motion
is inertial. In 2-4, he says the object is subject to a nonvanishing net force,
so its motion is noninertial. This matches up with the results of the energy analysis.

(c) The equivalence principle, as discussed on page \pageref{sec:chiao-paradox},
is vague, and is particularly difficult to apply successful and unambiguously to
situations involving electrically charged objects, due to the difficulty of
defining locality. Applying the equivalence principle in the most naive way,
we predict that there can be no radiation in cases 2 and 3 (because the object is
following a geodesic, minding its own business).
In case 4, everyone agrees that there will be radiation observable back on earth
(although it's possible that it would not be observable to an observer momentarily
matching velocities with the rocket).
The naive equivalence principle says that 1 and 4 must give the same result, so
we should have radiation in 1 as well. These predictions are wrong in two out of
the four equations, which tells us that we had better either not apply the equivalence
principle to charged objects, or not apply it in such a naive way.

\hwsolnhdr{dewitt-estimate}

(a) The dominant form of radiation from the orbiting charge will be the lowest-order
nonvanishing multipole, which in this case is a dipole. The power radiated from
a dipole scales like $d^2\omega^4$, where $d$ is the dipole moment. For an orbit of
radius $r$, this becomes $q^2r^2\omega^4$. To find the reaction force on the charged particle,
we can use the relation $p=E/c$ for electromagnetic waves (section \ref{sec:gravitational-redshifts}),
which tells us that the force is equal to the power, up to a proportionality constant $c$.
Therefore $a_r\propto q^2r^2\omega^4/m$. The gravitational acceleration is $a_g=\omega^2 r$,
so we have $a_r/a_g \propto (q^2/m)\omega^2 r$, or $a_r/a_g \propto (q^2/m)a_g$, where the
$a_g$ on the right can be taken as an orbital parameter, and for a low-earth orbit is very nearly equal to
the usual acceleration of gravity at the earth's surface.

(b) In SI units, $a_r/a_g \sim (k/c^4)(q^2/m)a_g$, where $k$ is the Coulomb constant.

(c) The result is $10^{-34}$. If one tried to do this experiment in reality, the effect would be
impossible to detect, because the proton would be affected much more strongly by ambient electric and
magnetic fields than by the effect we've calculated.

Remark: It is odd that the result depends on $q^2/m$, rather than on the charge-to-mass ratio $q/m$,
as is usually the case for a test particle's trajectory. This means that we get a different answer
if we take two identical objects, place them side by side, and consider them as one big object! This is not as unphysical
as it sounds. The two side-by-side objects radiate coherently, so the field they radiate is doubled, and the
radiated power is quadrupled. Each object's rate of orbital decay is doubled, with the extra effect coming
from electromagnetic interactions with the other object's fields.

\beginsolutions{2}

\hwsolnhdr{clock-postulate}

(a) Let $t$ be the time taken in the lab frame for the light to go from one mirror to the other,
and $t'$ the corresponding interval in the clock's frame. Then $t'=L$, and $(vt)^2+L^2=t^2$,
where the use of the same $L$ in both equations makes use of our prior knowledge that there
is no transverse length contraction.
Eliminating $L$, we find the expected expression for $\gamma$, which is independent of $L$
(b) If the result of a were independent of $L$, then the relativistic time dilation would depend
on the details of the construction of the clock measuring the time dilation. We would be forced
to abandon the geometrical interpretation of special relativity.
(c) The effect is to replace $vt$ with $vt+at^2/2$ as the quantity inside the parentheses
in the expression $(\ldots)^2+L^2=t^2$. The resulting correction terms are of higher order in
$t$ than the ones appearing in the original expression, and can therefore be made as small
in relative size as desired by shortening the time $t$. But this is exactly what happens when
we make the clock sufficiently small.

\hwsolnhdr{sagnac-area}

(a) Reinterpret figure \figref{thomas-as-area} on p.~\pageref{fig:thomas-as-area} as a picture of a Sagnac
ring interferometer. Let light waves 1 and 2 move around the loop in opposite senses. Wave 1 takes time
$t_{1i}$ to move inward along the crack, and time $t_{1o}$ to come back out. Wave 2 takes times
$t_{2i}$ and $t_{2o}$. But $t_{1i}=t_{2i}$ (since the two world-lines are identical), and similarly
$t_{1o}=t_{2o}$. Therefore creating the crack has no effect on the interference between 1 and 2,
and splitting the big loop into two smaller loops merely splits the total phase shift between them.
(b) For a circular loop of radius $r$, the time of flight of each wave is proportional to $r$, and
in this time, each point on the circumference of the rotating interferometer travels a distance
$v(\text{time})=(\omega r)(\text{time})\propto r^2$. (c) The effect is proportional to area, and
the area is zero. (d) The light clock in c has its two ends synchronized according to the Einstein
prescription, and the success of this synchronization verifies Einstein's assumption of commutativity
in this particular case. If we make a Sagnac interferometer in the shape of a triangle, then the Sagnac
effect measures the failure of Einstein's assumption that all three corners can be synchronized
with one another.

\hwsolnhdr{velocity-addition-matrix-taylor}

Here is the program:
\begin{listing}{1}
L1:matrix([cosh(h1),sinh(h1)],[sinh(h1),cosh(h1)]);
L2:matrix([cosh(h2),sinh(h2)],[sinh(h2),cosh(h2)]);
T:L1.L2;
taylor(taylor(T,h1,0,2),h2,0,2);
\end{listing}
The diagonal components of the result are both $1+\eta_1^2/2+\eta_2^2/2+\eta_1\eta_2+\ldots$
Everything after the 1 is nonclassical. The 
off-diagonal components are $\eta_1+\eta_2+\eta_1\eta_2^2/2+\eta_2\eta_1^2/2+\ldots$,
with the third-order terms being nonclassical.

\beginsolutions{3}

\hwsolnhdr{carousel-paradox}

The process that led from the Euclidean metric of example \ref{eg:metric-in-polar-coords} on page \pageref{eg:metric-in-polar-coords}
to the non-Euclidean one of equation [\ref{eq:rotating-spatial}] on page \pageref{eq:rotating-spatial} was not just a series
of coordinate transformations. At the final step, we got rid of the variable $t$, reducing the number of dimensions by one.
Similarly, we could take a Euclidean three-dimensional space and eliminate all the points except for the ones on the surface
of the unit sphere; the geometry of the embedded sphere is non-Euclidean, because we've redefined geodesics to be lines that
are ``as straight as they can be'' (i.e., have minimum length) while restricted to the sphere. In the example of the carousel, the final step effectively
redefines geodesics so that they have minimal length as determined by a chain of radar measurements.

\hwsolnhdr{carousel-metric}

(a) The $\der\theta'^2$ term of the metric blows up here. A geodesic connecting point A, at $r=1/\omega$, with point B, at $r<1/\omega$,
must have minimum length. This requires that the geodesic be directly radial at A, so that $\der\theta'=0$; for if not, then we could
vary the curve slightly so as to reduce $|\der\theta'|$, and the resulting increase in the $\der r^2$ term would be negligible
compared to the decrease in the $\der\theta'^2$ term. (b) The spatial track of a laser beam is nor a geodesic of this metric.
For example, a laser beam sent outward from the axis would make a track that was straight in the lab frame, but curved in the
rotating frame. Since the spatial metric in the rotating frame is symmetric with respect to clockwise and counterclockwise,
the metric can never result in geodesics with a specific handedness.

\beginsolutions{4}

\hwsolnhdr{lhc-proton-speed}

To avoid loss of precision in numerical operations like subtracting $v$ from $1$,
it's better to derive an ultrarelativistic approximation. The velocity corresponding
to a given $\gamma$ is $v=\sqrt{1-\gamma^{-2}}\approx 1-1/2\gamma^2$, so
$1-v\approx 1/2\gamma^2=(m/E)^2/2$. Reinserting factors of $c$ so as to make the units
come out right in the SI system, this becomes $(mc^2/E)^2/2=9\times 10^{-9}$.

\hwsolnhdr{dirac-sea-invariant}

Such a transformation would take an energy-momentum four-vector $(E,\vc{p})$, with $E>0$, to
a different four-vector $(E',\vc{p}')$, with $E'<0$. That transformation would also have the
effect of transforming a timelike displacement vector from the future light cone to the past
light cone. But the Lorentz transformations were specifically constructed so as to preserve
causality (property L5 on p.~\pageref{sec:lorentz-geometry}), so this can't happen.

\hwsolnhdr{doppler-three-d}

A spatial plane is determined by the light's direction of propagation and the relative velocity
of the source and observer, so the 3+1 case reduces without loss of generality to 2+1 dimensions.
The frequency four-vector must be lightlike, so its most general possible form is
$(f,f\cos\theta,f\sin\theta)$, where $\theta$ is interpreted as the angle between the direction of
propagation and the relative velocity. Putting this through a Lorentz boost along the $x$ axis,
we find $f'=\gamma f(1+v\cos\theta)$, which agrees with Einstein's equation on page
\pageref{einstein-doppler}, except for the arbitrary convention involved in defining the sign of $v$.

\beginsolutions{6}

\hwsolnhdr{carousel-singularities} 
(a) There are singularities at $r=0$, where $g_{\theta'\theta'}=0$, and $r=1/\omega$,
where $g_{tt}=0$. These are considered singularities because the inverse of the metric
blows up. They're coordinate singularities, because they can be removed by a change of
coordinates back to the original non-rotating frame.\\
(b) This one has singularities in
the same places. The one at $r=0$ is a coordinate singularity, because at small $r$
the $\omega$ dependence is negligible, and the metric is simply that of ordinary
plane polar coordinates in flat space. The one at $r=1/\omega$ is not a coordinate
singularity. The following Maxima code calculates its scalar curvature $R=R\indices{^a_a}$,
which is esentially just the Gaussian curvature, since this is a two-dimensional space.
\begin{listing}{1}
load(ctensor);
dim:2;
ct_coords:[r,theta];
lg:matrix([-1,0],
          [0,-r^2/(1-w^2*r^2)]);
cmetric();   
ricci(true);
scurvature();
\end{listing}
% carousel-singularities.mac
The result is $R=6\omega^2/(1-2\omega^2r^2+\omega^4r^4)$.
This blows up at $r=1/\omega$, which shows that this is not a coordinate
singularity. The fact that $R$ does not blow up at $r=0$ is consistent with our
earlier conclusion that $r=0$ is a coordinate singularity, but would not have been
sufficient to prove that conclusion.\\
(c) The argument is incorrect. The Gaussian curvature is not just proportional to
the angular deficit $\epsilon$, it is proportional to the
limit of $\epsilon/A$, where $A$ is the area of the triangle. The area of the triangle
can be small, so there is no upper bound on the ratio $\epsilon/A$.
Debunking the argument restores consistency with the answer to part b.
