\refstepcounter{appendixctr}\label{hwansappendix}%
\appendix\chapter{Appendix \ref{hwansappendix}: Hints and Solutions}
	
%==================================================================
%==================================================================
%========================= Solutions ==============================
%==================================================================
%==================================================================

\hwanssection{Solutions to Selected Homework Problems}

\beginsolutions{1}

\hwsolnhdr{ordered-geom-finite-models}

Pick two points P1 and P2. By O2, there is another point P3 that is distinct
from P1 and P2. (Recall that the notation [ABC] was defined so that all three
points must be distinct.) Applying O2 again, there must be a further point
P4 out beyond P3, and by O3 this can't be the same as P1. Continuing in this
way, we can produce as many points as there are integers.

\hwsolnhdr{have-spacesuit}

(a) If the violation of (1) is tiny, then of course Kip won't really have any
practical way to violate (2), but the idea here is just to illustrate the
idea, so to make things easy, let's imagine an unrealistically large violation
of (1). Suppose that neutrons have about the same inertial mass as protons, but
zero gravitational mass, in extreme violation of (1). This implies that neutron-rich
elements like uranium would have a much lower gravitational acceleration on earth
than ones like oxygen that are roughly 50-50 mixtures of neutrons and protons.
Let's also simplify by making a second unrealistically extreme assumption: let's
say  that Kip has a keychain in his pocket made of neutronium, a substance composed of
pure neutrons. On earth, the keychain hovers in mid-air. Now he can release
his keychain in the prison cell. If he's on a planet, it will hover.
If he's in an accelerating spaceship, then the keychain will follow Newton's
first law (its tendency to do so being measured by its nonzero inertial mass),
while the deck of the ship accelerates up to hit it.

(b) It violates O1. O1 says that objects prepared in identical inertial states
(as defined by two successive events in their motion) are predicted to have
identical motion in the future. This fails in the case where Kip releases the
neutronium keychain side by side with a penny.

\beginsolutions{2}

\hwsolnhdr{velocity-addition-matrix-taylor}

Here is the program:
\begin{listing}{1}
L1:matrix([cosh(h1),sinh(h1)],[sinh(h1),cosh(h1)]);
L2:matrix([cosh(h2),sinh(h2)],[sinh(h2),cosh(h2)]);
T:L1.L2;
taylor(taylor(T,h1,0,2),h2,0,2);
\end{listing}
The diagonal components of the result are both $1+\eta_1^2/2+\eta_2^2/2+\eta_1\eta_2+\ldots$
Everything after the 1 is nonclassical. The 
off-diagonal components are $\eta_1+\eta_2+\eta_1\eta_2^2/2+\eta_2\eta_1^2/2+\ldots$,
with the third-order terms being nonclassical.

\beginsolutions{3}

\hwsolnhdr{carousel-paradox}

The process that led from the Euclidean metric of example \ref{eg:metric-in-polar-coords} on page \pageref{eg:metric-in-polar-coords}
to the non-Euclidean one of equation [\ref{eq:rotating-spatial}] on page \pageref{eq:rotating-spatial} was not just a series
of coordinate transformations. At the final step, we got rid of the variable $t$, reducing the number of dimensions by one.
Similarly, we could take a Euclidean three-dimensional space and eliminate all the points except for the ones on the surface
of the unit sphere; the geometry of the embedded sphere is non-Euclidean, because we've redefined geodesics to be lines that
are ``as straight as they can be'' (i.e., have minimum length) while restricted to the sphere. In the example of the carousel, the final step effectively
redefines geodesics so that they have minimal length as determined by a chain of radar measurements.

\hwsolnhdr{carousel-metric}

(a) The $\der\theta'^2$ term of the metric blows up here. A geodesic connecting point A, at $r=1/\omega$, with point B, at $r<1/\omega$,
must have minimum length. This requires that the geodesic be directly radial at A, so that $\der\theta'=0$; for if not, then we could
vary the curve slightly so as to reduce $|\der\theta'|$, and the resulting increase in the $\der r^2$ term would be negligible
compared to the decrease in the $\der\theta'^2$ term. (b) The spatial track of a laser beam is nor a geodesic of this metric.
For example, a laser beam sent outward from the axis would make a track that was straight in the lab frame, but curved in the
rotating frame. Since the spatial metric in the rotating frame is symmetric with respect to clockwise and counterclockwise,
the metric can never result in geodesics with a specific handedness.

\beginsolutions{6}

\hwsolnhdr{carousel-singularities} 
(a) There are singularities at $r=0$, where $g_{\theta'\theta'}=0$, and $r=1/\omega$,
where $g_{tt}=0$. These are considered singularities because the inverse of the metric
blows up. They're coordinate singularities, because they can be removed by a change of
coordinates back to the original non-rotating frame.\\
(b) This one has singularities in
the same places. The one at $r=0$ is a coordinate singularity, because at small $r$
the $\omega$ dependence is negligible, and the metric is simply that of ordinary
plane polar coordinates in flat space. The one at $r=1/\omega$ is not a coordinate
singularity. The following Maxima code calculates its scalar curvature $R=R\indices{^a_a}$,
which is esentially just the Gaussian curvature, since this is a two-dimensional space.
\begin{listing}{1}
load(ctensor);
dim:2;
ct_coords:[r,theta];
lg:matrix([-1,0],
          [0,-r^2/(1-w^2*r^2)]);
cmetric();   
ricci(true);
scurvature();
\end{listing}
% carousel-singularities.mac
The result is $R=6\omega^2/(1-2\omega^2r^2+\omega^4r^4)$.
This blows up at $r=1/\omega$, which shows that this is not a coordinate
singularity. The fact that $R$ does not blow up at $r=0$ is consistent with our
earlier conclusion that $r=0$ is a coordinate singularity, but would not have been
sufficient to prove that conclusion.\\
(c) The argument is incorrect. The Gaussian curvature is not just proportional to
the angular deficit $\epsilon$, it is proportional to the
limit of $\epsilon/A$, where $A$ is the area of the triangle. The area of the triangle
can be small, so there is no upper bound on the ratio $\epsilon/A$.
Debunking the argument restores consistency with the answer to part b.
